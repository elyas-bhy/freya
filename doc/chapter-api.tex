\chapter{API REST}

L’un des avantages de GAE est qu’il permet de facilement mettre en place une API
REST et de l’exposer, grâce au module Google Cloud Endpoints
(GCE)\footnote{GCE:
\url{https://developers.google.com/appengine/docs/java/endpoints/}}.

Pour ce faire, on a simplement à annoter une classe comme étant une API, et
déclarer les méthodes d’API (notamment routes, paramètres, méthodes HTTP).\\

\begin{adjustbox}{minipage=1.02\textwidth,margin=0pt \smallskipamount,center}
\begin{lstlisting}[style=Java, label=endpoints, caption=Exemple d'API avec
Google Cloud Endpoints]
@Api(name = "myapi", version = "v1")
public class MyEndpoints {

	@ApiMethod(
	    name = "elements.get",
	    path = "elements/{element_id}",
	    httpMethod = HttpMethod.GET
	)
	public Element getElement(@Named("element_id") String id) {
	    // ...
	}
}
\end{lstlisting}
\end{adjustbox}


GCE se chargera ensuite de construire et déployer l’API avec la description des
méthodes exposées.\\
La description de l’API Freya (v1) se situe à cette
adresse\footnote{Descriptif:
\url{https://github.com/elyas-bhy/freya/wiki/freya-RESTful-API}}.\\
La version déployée de cette API se situe à cette adresse\footnote{Version
déployée de l'API: \url{http://goo.gl/o9GGBB}}.\\

Exemple d’interaction avec l’API en utilisant curl:
\begin{verbatim}
curl --header “Content-Type: application/json” -X GET
https://freya-app.appspot.com/_ah/api/freya/v1/artworks
\end{verbatim}
