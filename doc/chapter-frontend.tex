\chapter{Front-end}

Pour mettre en place le front-end de Freya\footnote{Front-end Freya:
\url{http://freya-app.appspot.com/}}, nous avons voulu profiter de la
bibliothèque client générée pour pouvoir accéder à l’API directement, sans
passer par des étapes intermédiaires. C’est la raison pour laquelle nous avons
choisi d’utiliser des pages JSP plutôt qu’utiliser GWT, qui est assez lourd à
mettre en place et très verbeux.

\section{Site web}

Le front-end de notre projet est un site Internet constitué de pages JSP. Ce
site est organisé de la manière suivante:\\

- A la racine du site (\emph{webapp}), un fichier \emph{index.jsp} sert de point
d’entrée par défaut dans l’application.\\

- Un dossier \emph{includes} regroupe les pages qui ne sont pas destinées à être
affichées seules mais qui sont là pour factoriser du code, notamment les parties
statiques du site (header, footer), mais aussi les manières de lister des
éléments.\\

- Pour chaque entité, un dossier contenant les fichiers \emph{view.jsp},
\emph{edit.jsp} et \emph{list.jsp} qui leurs sont propres. Le fichier
\emph{view} permet d’accéder à la vue détaillée d’une entité, permettant de
consulter l’ensemble de ses champs. Le fichier \emph{edit} permet de créer et
modifier une entité en renseignant ses champs. Le fichier \emph{list} permet de
lister l’ensemble des entités d’un même type. Pour cela il utilise le fichier à
inclure sachant lister une sous-partie ou l’ensemble de ces entités.

\newpage
\section{JavaScript}


Nous avons eu recours à plusieurs bibliothèques JavaScript, notamment JQuery et
ses modules DataTables, Chosen et Tabs.
Nous avons choisi ces plugins afin de rendre l’interface plus agréable à
utiliser.\\

Le plugin DataTables nous permet d’ajouter beaucoup de fonctionnalités à nos
tableaux. Notamment la pagination des données, et la possibilité de trier le
tableau suivant les colonnes.
De plus, DataTables peut être appliqué sur un tableau HTML existant, ou
encore nous pouvons lui passer en entrée des données en JSON, et lui demander de
construire la table lui-même.\\

Chosen est un plugin qui s’applique aux menus déroulants, permettant d’effectuer
des recherches au sein des menus, tout en appliquant une modification esthétique
agréable.\\

Le plugin Tabs est un outil de tabulation de div. Avec ce plugin nous avons pu
rendre plus agréable la disposition des pages présentant plusieurs tables. Avec
ce plugin, nous pouvons également spécifier, à l’aide d’un suffixe à l’URL, la
tabulation à afficher en priorité au moment du chargement de la page.
