\chapter{Bibliothèque client}

Un autre avantage de GCE est qu’il permet de générer automatiquement des
bibliothèques clients multi-plateformes (Java/Android, iOS, JavaScript) afin de
faciliter l’interaction avec l’API.
La bibliothèque client consiste en deux parties différentes:\\

\textbf{Le modèle client}: représentation en Java des entités manipulables. Ces
classes représentent le modèle existant du backend. Elles étendent la classe GenericJson
afin d’être sérialisable sur le réseau.
Exemple: le client dispose de la classe Artist (et son agrégat
ArtistCollection\footnote{Des classes de type agrégats sont générés aussi car on
peut sérialiser que des objets de type “bean”, c.à.d avec des getters/setters.
Les objets de type List<Object> ne sont pas conformes et doivent donc être
encapsulés dans une classe ObjectCollection.}), générée par GCE. Cette classe
est une représentation fidèle à celle du backend.\\

\textbf{Builders et stubs}: servent à faciliter la communication avec le
backend.
La classe générée Freya (nom de l’application) met en place une interface facile
à utitliser pour le client afin d’exécuter des méthodes de l’API.\\


\begin{adjustbox}{minipage=0.92\textwidth,margin=0pt \smallskipamount,center}
\begin{lstlisting}[style=Java, label=endpoints, caption=Exemple d'utilisation
de la bibliothèque client]
Artwork a = freya.artworks().get(artworkID).execute();
\end{lstlisting}
\end{adjustbox}