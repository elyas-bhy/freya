\chapter{Présentation}
Freya est une application web conçue pour faciliter la gestion des musées aux
conservateurs.\\
Ce projet a été mis en place grâce à de nombreux outils et frameworks, notamment
Maven\footnote{Maven: \url{http://maven.apache.org/}} et Google App
Engine\footnote{Google App Engine:
\url{https://developers.google.com/appengine/}}.


\section{Maven}
Maven est un outil de gestion de projet. Il permet notamment de spécifier
comment un logiciel doit être compilé, et décrire ses dépendances. Il permet
aussi d’intégrer des plugins dans des phases de compilation spécifiques, ce qui
permet d’avoir un environnement configurable.

\section{Google App Engine}
Google App Engine (GAE) est une PaaS (Platform as a Service) qui fournit un
ensemble d’outils afin de faciliter le développement d’applications web. Cette
plateforme met à disposition du développeur un runtime Java, une base de données
NoSQL (appellée \emph{datastore}), et l’hébergement de l’application web sur les
serveurs Google.\\

Afin d’intéragir avec le datastore, on peut utiliser plusieurs méthodes
différentes, comme par exemple l’API Datastore de bas niveau, JPA, JDO, ou
encore d’autres bibliothèques tierces comme Objectify.
Freya a été conçu en utilisant l’implémentation JPA 2.0 de
DataNucleus\footnote{DataNucleus:
\url{https://code.google.com/p/datanucleus-appengine/}} pour sa compatibilité avec GAE.\\

GAE fournit aussi une console d’administration très complète pour monitorer
l’état de l’application web, le traffic, les requêtes, les logs, etc...
